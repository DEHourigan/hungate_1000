\documentclass[10pt]{article}

\usepackage{texshade}

\headheight=0pt
\headsep=0pt
\hoffset=0pt
\voffset=0pt
\paperwidth=20in
\paperheight=9in
\ifx\pdfoutput\undefined
\relax
\else
\pdfpagewidth=\paperwidth
\pdfpageheight=\paperheight
\fi
\oddsidemargin=-0.9in
\topmargin=-0.7in
\textwidth=19.8in
\textheight=8.4in

\pagestyle{empty}

\begin{document}
\begin{texshade}{core_peptide_alignments/lassopeptide/lassopeptide.fasta}
\seqtype{P}
\shadingmode{identical}
\threshold{10}
\showconsensus[ColdHot]{bottom}
\shadingcolors{blues}
\showsequencelogo[rasmol]{top}
\hidelogoscale
\shownames{left}
\nameseq{1}{DKDFGO 05825 Lasso RiPP family leader peptide-containing protein}
\nameseq{2}{MDCENI 12925 Lasso RiPP family leader peptide-containing protein}
\nameseq{3}{APNHLP 03785 Lasso peptide}
\nameseq{4}{MDDLCH 11040 Lasso peptide}
\nameseq{5}{EGPPOI 06395 Paeninodin family lasso peptide}
\nameseq{6}{BAPACM 11765 Paeninodin family lasso peptide}
\nameseq{7}{NLCFLM 07155 Paeninodin family lasso peptide}
\nameseq{8}{GKGCKK 07220 Paeninodin family lasso peptide}
\nameseq{9}{PPECAL 02995 hypothetical protein}
\nameseq{10}{HMHFPH 06130 Paeninodin family lasso peptide}
\nameseq{11}{PHFKJA 13800 Paeninodin family lasso peptide}
\nameseq{12}{GMAAKK 18210 Paeninodin family lasso peptide}
\nameseq{13}{MOBKGA 10540 hypothetical protein}
\nameseq{14}{IDKLLN 15895 Paeninodin family lasso peptide}
\nameseq{15}{CKFMIE 00455 Paeninodin family lasso peptide}
\nameseq{16}{CKFMIE 00465 Paeninodin family lasso peptide}
\nameseq{17}{CKFMIE 00470 Paeninodin family lasso peptide}
\hidenumbering
\showlegend
\end{texshade}
\end{document}
